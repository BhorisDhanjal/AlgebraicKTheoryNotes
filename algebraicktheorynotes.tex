\documentclass[12pt]{article}
\usepackage{graphicx}
\usepackage{amsmath}
\usepackage{fontawesome5}
\usepackage{booktabs}
\usepackage{amssymb}
\usepackage{amsthm}
\usepackage{lmodern}
\usepackage[english]{babel}
\usepackage[utf8x]{inputenc}
\usepackage[toc,page]{appendix}
\usepackage[nottoc]{tocbibind}
\numberwithin{equation}{section}
\graphicspath{ {./Images/} }
\usepackage[raggedright]{titlesec}
\usepackage{placeins}
\usepackage{tikz}
\usepackage{mathtools}
\usepackage{float}
\usepackage[autostyle]{csquotes}\usepackage{quiver}
\usepackage[activate={true,nocompatibility},final,tracking=true,kerning=true,spacing=true,factor=1100,stretch=10,shrink=10]{microtype}
\usepackage{hyperref}

\newcommand{\R}{\mathbb{R}}
\newcommand{\Q}{\mathbb{Q}}
\newcommand{\C}{\mathbb{C}}
\newcommand{\Z}{\mathbb{Z}}
\newcommand{\N}{\mathbb{N}}
\newcommand{\F}{\mathbb{F}}
\newcommand{\Hom}{{\mathrm{Hom}}}
\newcommand{\image}{{\mathrm{Im}}}
\newcommand{\kernel}{{\mathrm{Ker}}}
\newtheorem{theorem}{Theorem}[section]
\newtheorem{definition}{Definition}[section]
\newtheorem{corollary}{Corollary}[theorem]
\newtheorem{lemma}[theorem]{Lemma}

\newtheorem{proposition}{Proposition}[section]
%opening
\title{Algebraic K-Theory}
\author{Bhoris Dhanjal}
\begin{document}
	\tableofcontents
	\maketitle
	\section{Small K groups}
	The category of finitely generated projective modules is the main object of study in algebraic K-theory. This is largely motivated by the following theorem due to Swan \cite{Swan1962} which relates algebraic K-theory to topological K-theory.
	\begin{theorem}[Swan's theorem]
		There exists an equivalence of categories between $\mathrm{Vect}(X)$ the category of vector bundles over a compact, Hausdorff space $X$ and finitely generated projective $C(X)$ modules. With the cross section functor.
	\end{theorem}
	\begin{proof}
		The functor we use is sending a vector bundle to its set of sections which takes on a natural module structure over $C(X)$. To show that it infact goes to f.g. proj modules 
	\end{proof}
	
	\subsection{Grothendieck group $K_0$}
	The big picture idea that Grothendieck had was that of a free completion of a commutative monoid. Commutative monoids occurred in nature very often as finitely generated projective modules/vector bundles. 
	
	This is a fairly natural approach which results in a Free-Forgetful adjoint pair between $\mathrm{CMon}$ and $\mathrm{Ab}$. We will refer to Weibel for most of the definitions \cite{weibel2013k}
		
	\begin{proposition}[Group completion functor]
		Assign $(A,+) \in \mathrm{CMon} $ to \[K_0(A) \in \mathrm{Grp}\] by taking the free group on symbols $[a]$ for $a \in A$ and quotienting the monoidal relations $[m+n]-[m]-[n]$.
	\end{proposition}
	
	The mapping is an injection iff the monoid is cancellative.
	
	\begin{proposition}[Mayer-Vietoris for group completions]
		content...
	\end{proposition}
	
	\begin{definition}[$K_0$ for a ring $A$]
		Consider the isomorphism classes of finitely generated projective modules over $A$. This forms a commutative monoid so consider its group completion $K_0(A)$
	\end{definition}
	
	\begin{proposition}[Eilenberg Swindle]
		$K_0$ for many abelian categories are trivial. If we consider $R^\infty$ as a infinitely generated free module over a ring $R$ if $P \oplus Q \equiv R^n$ then \[ P \oplus R^\infty \cong P \oplus (Q \oplus P) \oplus (Q \oplus P) \oplus \dots \equiv (P \oplus Q) \oplus (P \oplus Q) \oplus \dots \equiv R^\infty \] but this relation would imply $[P]=0 $ for all projectives. 
	\end{proposition}
	This extends to higher K groups with an analogue that demostrates the Quillen K space contracts, see V.1.9 in \cite{weibel2013k}.
	
	\begin{definition}[Morita equivalence for rings and ]
		content...
	\end{definition}
	
	\begin{proposition}\label{k0pidisZ}
	If $A$ is a Field/local ring/PID then $K_0(A)=\Z$
	\end{proposition}
	\begin{proof}
		For fields and division rings its just due to all f.g. modules being equal to some $A^n$. Similarly as seen in \ref{a2} f.g. proj. modules in a local ring are free. And finally using the structure theorem for a PID and the fact that f.g. proj. modules are direct summands of a f.g. free module which being a submodule of a free module is free (refer to \ref{submodoffreemodisfreepid}).
		
		So in each case $\mathrm{Proj}(A) \cong \N$ so its group completion is $\Z.$
		\end{proof}
		
	\begin{lemma}
		$K_0(A) \cong \Z \implies $ projective modules over $A$ are stably free.
	\end{lemma}
	\begin{theorem}\label{extensionk0iscong}
		For a ring $A$ the map $A \to A[x]$ induces an isomorphism $K_0(A) \cong K_0(A[x])$ 
	\end{theorem}
	
	\begin{definition}[$K_0$ for abelian category $\mathcal{A}$]
		$K_(\mathcal A)$ is generated by $[A]$ for each $A \in \mathcal{A}$ and a relation of $[A]=[A']+[A'']$ for all \[ 0 \to A' \to A \to A'' \to 0 \]
		short exact
	\end{definition}
	\begin{definition}[Relative $K_0$ groups]
	\end{definition}
	\subsection{Whitehead group $K_1$}
	\begin{definition}[Whitehead group for a ring]
		$K_1= \frac{GL(A)}{[GL(A):GL(A)]}$
		Where $GL(A)$ denotes the colimit of $GL_n(A)$ with $GL_{n}$ realized as a subgroup of $GL_{n+1}$ by placing the matrix in the top left corner. 
	\end{definition}
	
	\begin{lemma}[Whitehead]
		content...
	\end{lemma}
	\begin{proposition}
		\[ [GL(A):GL(A)]=E[A] \]
	\end{proposition}
	\begin{proof}
		content...
	\end{proof}	
	
	\begin{definition}
		$SK_1(A):= \ker \det$
		
		Where, $\det : K_1(A) \to A^*$. We have a split exact sequence
		\[ 0 \to SK_1(A) \to K_1(A) \to A^* \to 0 \]
	\end{definition}
	
	
	\begin{lemma}
		For E.D. $A$ we have $SL_n(A)=EL_n(A)$ for all $n.$
	\end{lemma}
	\begin{proof}
		With elementary row and column operations arrange the matrix so that the element with the smallest norm is in the top right position. And using elementary row operations reduce it to a matrix with a unit in the top left and 0s in the rest of the first column and first row. Proceeding similarly for the remaining $n-1 \times n-1 $ matrix left we reduce it down to a matrix of the form.
		
		\[ \begin{bmatrix}
			u_1 & 0 & \dots & 0 \\
			0 & u_2 & \dots & 0 \\
			\vdots & 0 & \ddots & 0\\
			0 & 0 & \dots & u_n 
		\end{bmatrix} \]
		
	Now apply Whiteheads lemma 
	\end{proof}
	
	\begin{theorem}[Suslin]
		$E_n(A)$ normal in $GL_n(A)$ for $n \geq 3$. 
	\end{theorem}
	\begin{proof}
		content...
	\end{proof}
	
	We will know move towards Horrock's theorem. We follow Lang's book for the first few results which recounts Vaserstein's proof of Quillen-Suslin \cite{lang02}.
	
	\begin{definition}
	An $A$ module $M$ is stably free if there exists a f.g. free module $F$ such that $M \oplus F$ is free.
	\end{definition}
	\begin{lemma}
	A proj. module is stably free iff if has a finite free resolution.
	\end{lemma}
	\begin{theorem}[Hilbert-Serre (try to avoid writing this will work with Th \ref{k0pidisZ} and Th \ref{extensionk0iscong})]
	Every f.g. module over $k[x_1,\dots x_n]$ is stably free.
	\end{theorem}
	
	
	\begin{definition}[Unimodular row]
		For a ring $A$, an element of $A^n$ is said to be a unimodular row if its components generate	$A$. 
	\end{definition}
	
	Alternatively it can be useful to view a unimodular row as as element of $M_{1 \times n} (A) $ as such it represents a surjective linear map $A^n \to A$, or even an element in $M_{n \times 1}$ in which case it represents a injection from $A \to A^n$.
	
	\begin{definition}[Equivalence of unimodular rows]
		For unimodular rows $v,w\in A^n$ we say $v \sim w $ if $\exists M \in GL_n(A)$ such that $Mv=w$.
	\end{definition}
	
	\begin{definition}[Unimodular extension property]
		Given a unimodular row $v=(v_1,\dots v_n) \in A^n$ if we can construct a invertible $n \times n $  matrix with $v$ in the first column we say $v$ has the unimodular extension property.
	\end{definition}
	
	We don't use the above terminology in the light of the following fact.
	
	\begin{lemma}
		A unimodular row $v \in A^n$ has the unimodular extension property iff $v \sim (1,0,\dots ,0)$
	\end{lemma}
	\begin{proof}
		If $v$ can be extended to a invertible matrix $M \in GL_n(A)$ then \[ M{^-1} = (1,0,\dots, 0) \].
		Conversely if $M' \in GL_n(A) $ s.t. $M'v=(1,0,\dots,0)$ then $M'{^-1}$ has $v$ in the first column.
	\end{proof}
	
	\begin{proposition}\label{inductionbaseforprequillensuslin}
		Over a PID $A$ any two unimodular rows in $A^n$ are equivalent.
	\end{proposition}
	\begin{proof}
		We will instead show that any unimodular row is equivalent to $(1,0,\dots, 0)$. Since its a PID one element must generate $A$ move this to the first coordinate by elementary row operations then make it equivalent to $1$ by another row operation.
	\end{proof}
	\begin{proposition}
		Over a local ring $A$ any two unimodular rows are equivalent
	\end{proposition}
	\begin{proof}
		Use the fact that projective modules over local rings are free.
	\end{proof}
	\begin{theorem}[Horrocks' theorem]
	If $(A, \mathfrak{m})$ is a local ring then for any arbitrary unimodular row $v(x)$ in $A[x]^n$ such that one of its component elements has leading coefficient 1 implies that $v$ has the unimodular extension property. Furthermore, any such $v$ is equivalent to $v(0)$.
	\end{theorem}
	\begin{proof}
	Recall that for a local ring $x \not \in \mathfrak m \iff x  $ is a unit.
	
	When $n=1,2 $ there is nothing to prove. Assume $n \geq 3$.
	
	
	Without loss of generality, we take $v_1(x)$ with degree $d $ among components with leading coefficient $1$ and $\deg v_i < d, $ for $i \neq 1$. We shall induct on $d$.
	
	By unimodularity we know there exists $w(x)\in A[x]^n$ such that,
	\[ \sum_{i=1}^n w_i v_i = 1 \]
	So we can say that not all of the coefficients of $v_2, \dots v_n $ can lie in $\mathfrak m$. For if it were the case, then reduced mod $\mathfrak m$ we arrive at a contradiction since we assumed $v_1 $ has leading coefficient 1 and $w_1v_1$ wouldn't have a constant residue.
	
	Once again without loss of generality, assume some coefficient of $v_2(x)$ does not lie in $\mathfrak m$, and as such is a unit.
	
	Now consider the ideal $I$ generated by the leading coefficients of $w_1v_1+w_2v_2$ of degree $< d.$ 
	
	$I$ contains the coefficients of $v_2$ this can be inductively found when $w_1=0, w_2=1$ we get the coefficient of the $x^m$ term where $\deg v_2 = m$.
	Using repeatedly different choices of polynomials we are done.
	
	Since $I$ has a unit which means it generates $A$. And consequently implies that there was some choice of polynomial $y_1v_1+y_2v_2$ of degree $<d$ with leading coefficient $1$.
	
	The the appropriate row actions we can obtain this in some component of $v$. Repeating this process until we get $d=0$ finishes the proof.
	
	Now because of $\sum_{i=1}^n w_i v_i =1 $ there must be some constant term not in $\mathrm m $ and unital as such. So $v(0) \sim (1,0,\dots ,0 ) \sim v$ as seen above.
	\end{proof}
	
	We now extend the idea of Horrock's theorem.
	
	\begin{lemma}\label{horrocksbutforlocal}
		For an integral domain $A$ and a multiplicative subset $S$ if $v(x) \sim v(0)$ over $A_S[x]^n $ then there exists $c \in S$ such that $v(x+cy) \sim v(x) $ over $A[x,y]^n$
	\end{lemma}
	\begin{proof}
		By the equivalence $v(x) \sim v(0)$ we know there exists a matrix $M \in GL_n(R_S[x])$ such that $M(x)v(x)=v(0) $ now consider \[ N(x,y) = M(x)^{-1} M(x+y) \]
		
		Note that now $N(x,y)v(x+y)=v(x)$ and so also $y \mapsto cy$ implies that $N(x,cy)v(x+cy)=v(x).$
		
		Now to show that indeed $N(x,cy)\in R[x,y]$ for some choice of $c \in S$ but this is true since $N(x,0)=I_N \implies N(x,y)=I+yP $ for some $P \in R_S[x,y]$ but this just means there is some appropriate choice of $c \in S$ that allow us to cancel out all the denominators in $P$ so that $P[x,cy] \in R[x,y]$.
	\end{proof}
	
	\begin{lemma}\label{horrocksbuteverything}
		For a ring $A$ and $v(x)$ unimodular row in $A[x]^n$ with at least one component having leading coefficient one implies $v(x) \sim v(0)$.
	\end{lemma}
	\begin{proof}
		Consider the set $I$ containing all $c \in A$ such that $v(x+cy)\sim v(x)$ as rows in $A[x,y]$ if the ideal contains $1$ then sending $x \to 0$ would give us $v(y)\sim v(0) $ in $A[y].$
		
		We can achieve this by first showing $I$ is an ideal and then showing that its not contained in any maximal ideal.	To do this last step we will localize at the maximals and use the previous result.
		
		First prove that $I$ is an ideal.
		\begin{enumerate}
			\item $I \neq \emptyset $ as $0 \in I$
			\item If $c,d \in I$ then $c-d \in I$ as $v(x+(c-d)y)=v(x+cy-dy) \sim v(x+cy) \sim v(x)$ by a substitution $x \mapsto x+cy$
			\item For $a \in A, c \in I$ then simply $v(x+cay) \sim v(x)$ by the $y \mapsto ay$
		\end{enumerate}
		
		Now to show $I$ isn't contained in any maximal ideal. Pick a maximal ideal $\mathfrak m$ and localize at it first due to Horrocks we know $v(x) \sim v(0) $ in $A_{\mathfrak m} [x]$ and then due to the previous lemma \ref{horrocksbutforlocal} we find some $c \in A\setminus \mathfrak m$ such that $v(x+cy) \sim v(x) \sim v(0)$ but this just means that $ c \in I$ and so $I\not \subset \mathfrak m$ this applies to any maximal and so we are done.  
	\end{proof}
	
	\begin{theorem}
		For $A=k[x_1, \dots, x_n]$ where $k $ is a PID, then $v \sim (1,0,\dots, 0)$ for any unimodular row $v \in A^n$.
	\end{theorem}
	\begin{proof}
		Proceed with induction on $n$. We proved $n=0$ above Prop. \ref{inductionbaseforprequillensuslin}.
		
		Assume $n\geq 1$ and that the result holds for $m-1$.
		
		Then $v \in k[x_1, \dots, x_m] \cong k[x_1,\dots, x_{m-1}] [x_m]$ can be realized as $v(x_m) $ with coefficients in $k[x_1,\dots, x_{m-1}]$. If $v(x_m)$ has some component with leading coefficient $1$ then by Lemma \ref{horrocksbuteverything} we now $v(x_m) \sim v(0) \in k[x_1, \dots, x_{m-1}]$ and we can reduce by induction.
		
		So if not by some appropriate change of variables as amongst $x_1, \dots, x_{m-1}$ in the form of $x_i \mapsto x_i-x_m^{p_i}$ for very large $p_i$'s this allows us obtain the leading coefficient in terms of $x_m$ to be 1 as needed.
	\end{proof}

	
	
	\begin{theorem}[Quillen-Suslin]
		F.g. proj. modules over $A=k[x_1,\dots,x_n]$ where $k$ is a PID are free.
	\end{theorem}
	\begin{proof}
		We know such f.g. proj. modules are stably free. And from above we know any unimodular row in $A$ is equivalent to $(1,0,\dots,0)$.
		
		That is to say given a f.g. proj. module $P$ which is stably free, i.e. $P \oplus R^{m_1} \cong R^{m_2}$ then $P$ is free.
		
		When $m_1=1$ this is the split exact sequence (since P is projective see \ref{projtfae}),
		\[ 0 \to A \hookrightarrow A^{m_2}  \twoheadrightarrow P \to 0 \]
		The injection $A \to A^{m_2}$ is precisely a unimodular row by definition which we know must correspond to the canonical embedding of $1 \mapsto (1,0,\cdots, 0)$.
		So,$$P = \mathrm{im}(A^{m_2} \to P) \cong A^{m_2}/\ker (A^{m_2} \to P) \cong A^{m_2}/\mathrm{im}(A \to A^{m_2}).$$
		But $A^{m_2}/\mathrm{im}(A \to A^{m_2})$ is free since $\mathrm{im}(A \to A^{m_2})$ is naturally free due to the embedding.
		
		When $m_1 \neq 1$ just take $(P \oplus A^{m_1-1}) \oplus A$.
	\end{proof}
	\subsection{$K_2$}
	
	\section{Higher K theory}
	\subsection{Quillen Q construction}
	\subsection{Waldhausen construction}
	
	
	\begin{appendices}
	\section{Projective modules}
	Recall a\textbf{ free module} of rank $n$ is one that is isomorphic to $n$ direct sums of its underlying ring. And homomorphisms from free modules to other modules are determined by the image of their generators, i.e. free objects are left adjoints to forgetful functors. \footnote{This holds in free monoids $\mathrm{Hom}_\mathbf{Mon}(F(X), M) \cong \mathrm{Hom}_\mathbf{Sets} (X, U(M))$ where $F(X)$ denotes the free monoid generated by elements from the set $X$ and $U(M)$ is the underlying set of a monoid $M$, refer to \cite[p. ~208]{Awodey} }
	
	A module $P$ is said to be \textbf{projective} if it satisfies the following lifting property, every morphism from $P$ to $N$ factors through an epi into $N$. Note that the lift need not be unique this is \textit{not} an UMP
	\[\begin{tikzcd}
		&& M \\
		\\
		P && N
		\arrow[two heads, from=1-3, to=3-3]
		\arrow[from=3-1, to=3-3]
		\arrow[dashed, from=3-1, to=1-3]
	\end{tikzcd}\]
	\begin{lemma}[Free modules are projective]
	\end{lemma}
	\begin{proof}
		Consider the preimages of images of basis of $P$ in $N$, that lie in $M$. Then map basis elements from $P$ into these preimages.
	\end{proof}
	\begin{proposition}[Equivalent definitions of projectivity]\label{projtfae}
		TFAE,
		\begin{enumerate}
			\item $P$ is projective.
			\item For all epi's between $M\twoheadrightarrow N$, the induced map $\Hom(P,g):\mathrm{Hom}(P,M) \to \mathrm{Hom}(P,N)$ sending $f \mapsto g \circ f$ for $g:M \to N$ and $f:P \to M$ is an epi.
			\item For some epi from a free module $F$ to $P$, $\mathrm{Hom}(P,F) \to \mathrm{Hom}(P,P)$ is an epi.
			\item There exists $Q$ s.t. $P \oplus Q$ is free
			\item Short exact sequences of the form $0 \to A \to B \to P \to 0$ split, i.e. isomorphic to another short exact where middle term is $A \oplus P$ \footnote{In general any epis into projective objects split (i.e. have an inverse).}
		\end{enumerate}
	\end{proposition}
	\begin{proof}
		$1 \iff 2$ is restatement of definitions.
		
		$2 \implies 3$ also just substitution.
		
		$3 \implies 4$ consider a map in the preimage of identity in $\Hom(P,P)$ which is a splitting (inverse) of the epi $F$ into $P$,
		\[\begin{tikzcd}
			& P \\
			\\
			F && P
			\arrow["g", shift left=3, two heads, from=3-1, to=3-3]
			\arrow[""{name=0, anchor=center, inner sep=0}, "f"', from=1-2, to=3-1]
			\arrow[""{name=1, anchor=center, inner sep=0}, "{\mathrm{Id}_P=g \circ f}", dashed, two heads, from=1-2, to=3-3]
			\arrow[shorten <=6pt, shorten >=6pt, Rightarrow, from=0, to=1]
		\end{tikzcd}\]
		Now we have a short exact sequence $0 \to \ker g \to F \to P \to 0$, and also $f\circ g $ is idempotent so it naturally admits a decomposition $F = \image(f \circ g) \oplus \kernel (f \circ g)$\footnote{For some idempotent $e$, $1-e$ is also an idempotent and images under these two mappings decompose any module, furthermore image of $1-e$ is just kernel of $e$}=$\image (g) \oplus \kernel (g)$ the first by the 1st isomorphism theorem and the second by $f $ being a mono.
		
		$4 \implies 2$ simply as $\hom (P \oplus Q,-) = \hom(P,-) \oplus \hom(Q,-)$
		
		$1 \iff 5$ should be clear from above.
		
	\end{proof}
	
	\begin{theorem}[Proj. fin. generated modules over local rings are free]\label{a2}
	\end{theorem}
	\begin{proof}
		pick a minimal set of generators and see its residue classes in $M/\mathfrak{m}M$ as the basis of it as a vector space over $R/\mathfrak{m}$.
		
		Now as for some free module $F, F=\varphi(M)\oplus K$ for some $K$ and some homomorphism $\varphi: M \to F$, (by defn of projective module), 	we get \[ M/\mathfrak{m}M \cong 	F/\mathfrak{m}F = (R/\mathfrak{m})^n\cong R^n\otimes R/\mathfrak{m} \cong F \otimes R/\mathfrak{m} \cong (\varphi(M)\oplus K) \otimes R/\mathfrak{m}\]
		
		Finally we get $M/\mathfrak{m}M \cong M/\mathfrak{m}M \oplus K/\mathfrak{m}K\implies K=\mathfrak{m}K \implies K=0$ by Nakayama
	\end{proof}
	This holds for not necessarily finitely generated modules too refer to \cite[Th.~2.5]{matsumura_1987}	.
	\begin{proposition} If $M$ is a finitely presented module over a Noetherian ring $R$ (prime ideals fin gen) then TFAE
		\begin{enumerate}
			\item $M$ is projective.
			\item $M$ localized at maximal ideals is free.
			\item A finite set of elements $\{x_i\}^n$ in $R$ generate $R$ such that $M[x_i^{-1}]$ is free over $R[x_i^{-1}]$.
		\end{enumerate}		
	\end{proposition}
	This proceeds just from the previous result.
	
	\begin{proposition}\label{submodoffreemodisfreepid}
		For a PID $A$ a submodule of a free module over $A$ is free.
	\end{proposition}
	\begin{proof}
		Let $F \cong \oplus_{i \in I} A_i$ be a free module over $A$. And suppose there exists a well ordering on the indexing $I$.
		
		Now consider $f_j : \oplus_{i< j} (A_i) \oplus A_j \cap M \to \R $ defined by sending elements $(a,b) \to b$ where $a \in \oplus_{i<j}A_i, b \in A_j$.
	\end{proof}
	%		Note If $R$ is local and $M$ is fin-gen projective module then $M$ is free, this is a consequence of Nakayama. As $M \oplus Q = R$ so if $R$ has maximal ideal $\mathfrak{m}$ then $M/\mathfrak{m}M$ is a vector space	over the field $R/\mathfrak{m}R$ and its basis lifts to minimal set of generators of $M$, consider $N=M/\sum R m_i$ and so $ N/ IN=M/(IM+\sum_i R m_i)=M/M=0\implies N=IN$ then apply typical Nakayama to get $N=0
	%		\implies M= \sum_i R m_i$, for $I$ an ideal inside the Jacobson radical of $R$
	%		
	%		So now to prove $1 \iff 2 $ consider a finitely presented module localized over 
	%		
	%		If $M,N$ finitely presented over $R$ and their localizations are isomorphic then theres some element of $f\in R-P$ such that $M[f^{-1}] \cong N[f^{-1}]$
	\section{Resolutions}
	Given a module $M$ its \textbf{left resolution} is given by the data of a exact sequence $(A_\bullet, \varphi_\bullet)$ into $M$ as such,
	\[ 	\dots \to A_1	\to A_0 \xrightarrow{\epsilon} M \to 0 \]
	where $\epsilon $ is called the \textbf{augmentation map}, if the exact sequence is free its a free resolution and such for projective. 
	
	If we have a cochain complex instead it forms a \textbf{right resolution} and if its elements are injective we call them injective resolutions.
	
%	\textbf{TO DO: KOSZKUL COMPLEX AND HILBERT SYZYGY}
%	We previously saw the definition of a bilinear map when discussing tensor products. Now consider $n$- linear maps for some map between $R$-Modules $M,P$. Repeated application of the tensor product still provides us with universal such module, $M^n \to \otimes_{i=1}^n M$.
%	
%	Also a map is called $n$-alternating if it vanishes when two of the arguments are the same. This also implies that sign changes when the arguments are interchanged. Also for a permutation the sign change changes to the sign of the permutation. Now when we require a notion of a universal $n$-linear alternating map we arrive to the definition of a wedge product.
%	
%	In particular for a specific $n$. There is a universal alternating $n$-linear map sending $M \to \Lambda^n M$. If $M$ is finitely generated by $r$ elements then $\Lambda M^n$ is a free module of rank $\binom{r}{n}$.
%	
%	\begin{definition}[Tensor algebra]
%		For a $R$-Module $M$ and $T^n M = \otimes^n M$ the tensor algebra is defined to be \[ T(M)=\oplus_{n=0}^\infty T^n M \]
%	\end{definition}
%	
%	
%	
%	\begin{definition}[Exterior algebra and wedge product]
%		For an $R$-Module $M$. Consider the Tensor algebra $T(M)=\oplus_n \otimes^n M	$. Consider the ideal $I$ spanned by elements $v \otimes v$ for $v \in T(M)$.
%		
%		The quotient of $T(V)/I := \Lambda(M) $ is called the exterior algebra and its product the wedge product.	It is universal with respect to multilinear alternating maps.
%		
%		For $v \Lambda^nM, w \in \Lambda^m M$ multiplication is defined as such,
%		\[ v \Lambda w = (-1)^{nm} w \Lambda v\]
%	\end{definition}
%	
%	
%	\begin{definition}[Koszkul complex]
%		For a $R$-module $M$ there is an associated chain complex of $n^\textbf{th}$ exterior algebras
%		\[ \Lambda^n M \to \Lambda^{n-1} M \to \dots \to \Lambda^0 M \cong M \]
%	\end{definition}
%	
%	
%	\begin{definition}[Regular sequence]
%		A sequence of elements $r_1,\dots,r_n$ is said to be a regular sequence in $R$ if $r_1$ is not a zero divisor of $R$, $r_2$ is not a zero divisor of $R/\langle r_1\rangle $, and so on
%	\end{definition}
%	\begin{proposition}
%		For a finite regular sequence $\{r_i\}_{i=1}^n$ of a ring $R$ the Koszkul complex forms the canonical free resolution of $R/\langle r_1,\dots,r_n\rangle$ of the form
%		\[ 0 \to R^{\binom{n}{n}} \to \dots \to R^{\binom{n}{1}} \to R \to R/\langle r_1,\dots, r_n \rangle \to 0\]
%	\end{proposition}
%	\begin{proof}
%		TO DO
%	\end{proof}
	
	
	\section{Vector bundles}
	More detailed exposition can be found in \cite{milnor1974characteristic}. We define the basics as needed for Swam's theorem. We understand all maps as continuous functions.
	\begin{definition}[Vector bundle]
		A real $n$ dimensional vector bundle is a triple $(E,p, B)$. Which consists of a continuous map $p:E \to B$ from the total space $E$ to the base space $B$. Such that for all $b \in B$, $F_b=p^{-1}(b)$ the fibre of $b$ has a real/complex vector space structure. Along with the following property of local trivialization
		\begin{enumerate}
			 \item For any $b \in B$ there exists a open open $U \subset B$ along with a homeomorphism \[ h: U \times \R^n	\to p^{-1}(U) \] such that for all $c \in U$ the map through $h$ defines an isomorphism between $F_c$ and $\R^n$.
		\end{enumerate}
	\end{definition}
	
	A trivial vector bundle is one in which the total space $E=B \times \R^n$ with $p$ just the trivial projection mapping.
	
	
	\begin{definition}[Vector bundle isomorphisms]
		Two vector bundles $(E_1,p, B)$ and $(E_2,p_2, B) $ are considered isomorphic if there exists a homeomorphism between their total spaces $h$ such that the below diagram commutes
		\[\begin{tikzcd}
			{E_1} && {E_2} \\
			& B
			\arrow["{p_1}"', from=1-1, to=2-2]
			\arrow["{p_2}", from=1-3, to=2-2]
			\arrow["h", from=1-1, to=1-3]
		\end{tikzcd}\]
		
		and also if $h$ induces a vector space isomorphism for each fibre.
	\end{definition}
	
	\begin{definition}[Sections of a vector bundle]
		For a topological vector bundle $(E,p,B)$ a section refers to a map	$s: B \to E$ such that $p \circ s= 1_B$ where $1_B$ denotes the identity map on $B$.
		
		These sections equivalently are a homomorphism of vector bundles from the trivial line bundle $(B \times \R, \pi_B, B) \to (E,p,B)$
	\end{definition}
%	
%	\begin{definition}[Pullbacks of bundles]
%		content...
%	\end{definition}
%	
%	\begin{definition}[Whitney sums]
%		content...
%	\end{definition}
%	
%	\begin{definition}[Steifel Whitney Class]
%		content...
%	\end{definition}
	\section{Categories}
	\subsection{Abelian Categories (shorten this too many example)}
	There is a chain of conditions regarding `abelian'-ness of categories which is roughly understood as follows,
	\[ \textbf{Abelian} \subseteq \textbf{Pre-Abelian} \subseteq \textbf{Additive} \subseteq \textbf{Ab-Enriched}\]
	The motivation behind them is to have categories which resemble algebras.
	
	Ab-Enriched categories are categories such that for objects $A,B \in \mathbf{C}$ the external hom set $\Hom(A,B)$ has the structure of an abelian group, furthermore it has a well defined notion of composition (which is bilinear due to the monoidal product in Ab), $\Hom(A,B)\otimes \Hom(B,C) =\Hom(A,C)$. 
	\begin{proposition}
		In Ab-Enriched categories intial and terminal objects coincide (it is often called the zero object)
	\end{proposition}
	\begin{proof}
		Let $\mathbf{C}$ be an Ab-Enriched category. Note that the Hom-sets between objects have `zero morphisms', i.e. arrows in the Hom-set which behave like the additive identity in the Ab group induced by it. In particular for $0_{A,B}\in \Hom(A,B)$ we have the property that if $f:B \to C$ then $f\circ 0_{A,B}=0_{A,C}$ and $g: A \to D$ then $0_{A,B}\circ g=0_{D,B}$.
		
		Now suppose $0 \in \mathbf{C}$ is initial so there is a unique morphism $0\to 0$ so in its Hom-set its both the additive inverse and the identity. So for any $f:X \to 0$ we can say that by the zero morphism property $f=0$ so also $0$ is terminal.
	\end{proof}
	\begin{proposition}
		In Ab-Enriched categories finite coproducts coincide with finite products (i.e. biproducts) \footnote{This also holds over categories enriched over commutative monoids.}
	\end{proposition}
	\begin{proof}	
		Let $\mathbf{C}$ be an Ab-enriched category and $A,B\in \mathbf{C}$ consider the product $A\times B$, which is determined by the following UMP,
		\[\begin{tikzcd}
			& X \\
			A & A\times B & B
			\arrow["{p_1}"', from=2-2, to=2-1]
			\arrow["{p_2}", from=2-2, to=2-3]
			\arrow["u", dashed, from=1-2, to=2-2]
			\arrow["{x_1}"', from=1-2, to=2-1]
			\arrow["{x_2}", from=1-2, to=2-3]
		\end{tikzcd}\]
		Consider $A$ and $B$ in place of $X $ in the diagram. By the UMP we have $q_1: A \to A\times B, q_2: B \to A\times B$
		\[\begin{tikzcd}
			A && B \\
			& {A\times B} \\
			A && B
			\arrow["{p_1}"', from=2-2, to=3-1]
			\arrow["{p_2}", from=2-2, to=3-3]
			\arrow["{q_2}"', from=1-3, to=2-2]
			\arrow["{q_1}"', from=1-1, to=2-2]
			\arrow["{1_A}"', from=1-1, to=3-1]
			\arrow["{1_B}", from=1-3, to=3-3]
		\end{tikzcd}\]
		So $p_1q_1=1_A$ and $p_2q_2=1_B$ also $p_1q_2=p_2q_1=0$.
		
		Now note that $q_1p_1+q_2p_2=1_{A\times B}$ as $p_1(q_1p_1+q_2p_2)=p_1$ and $p_2(q_1p_1+q_2p_2)=p_2$. Claim this $q_1,q_2$ determine a coproduct $A +B$.
		
		We wish to show the following UMP holds for some arbitrary $C \in \mathbf{C}$
		\[\begin{tikzcd}
			A & C & B \\
			& {A\times B} \\
			A && B
			\arrow["{q_1}"', from=1-1, to=2-2]
			\arrow["{q_2}", from=1-3, to=2-2]
			\arrow["{p_1}"', from=2-2, to=3-1]
			\arrow["{p_2}", from=2-2, to=3-3]
			\arrow["{1_B}"', from=1-3, to=3-3]
			\arrow["{1_A}"', from=1-1, to=3-1]
			\arrow["{r_1}", from=1-1, to=1-2]
			\arrow["{r_2}"', from=1-3, to=1-2]
			\arrow["{f}"',dashed, from=2-2, to=1-2]
		\end{tikzcd}\]
		Define $f: A\times B \to C$ as $f=r_1p_1+r_2p_2$. Now $fq_1=r_1$ and $fq_2=r_2$ if we show uniqueness of $f$ we are done.
		
		Say $f'$ then $(f-f')1_{A \times B}=(f-f')(q_1p_1+q_2p_2)=0$. So $f=f'$.
		
		
	\end{proof}
	
	
	\begin{definition}[Additive category]
		An Ab-Enriched category which has all finite coproducts.
	\end{definition}
	
	Functors between additive categories are called \textit{additive functors}. And can be realized as functors which preserve additivity of homomorphisms between modules, $F(f+g)=F(f)+F(g).$
	
	Before proceeding further it is important to think about kernels and cokernels in the categorical sense.
	\begin{definition}[Kernel]
		A kernel is a pullback of a morphism $f:A \to B$ and the unique morphism from $0 \to B$. Provided initials and pullbacks exist.
		
		\[\begin{tikzcd}
			{\ker f} && 0 \\
			\\
			A && B
			\arrow["f", from=3-1, to=3-3]
			\arrow[from=1-3, to=3-3]
			\arrow[from=1-1, to=1-3]
			\arrow[from=1-1, to=3-1]
		\end{tikzcd}\]
	\end{definition}
	The intuition behind this definition is that alternatively it is seen as an equalizer of a function $f:A \to B$ and the unique zero morphism $0_{A,B}$. The kernel object is the part of the domain that is 'going to zero'. \footnote{A minor point to note is that in the case of Ab-Enrichments the `zero' in the Hom-sets isn't a terminal, its Hom-set specific. When you assume a Ab-Enriched category has a initial 0 however this matches up with our intuition.}
	
	
	\begin{definition}[Pre-abelian categories]
		An additive category with all morphism having kernels and cokernels.
	\end{definition}
	The above definition is equivalent to saying a pre-abelian category is a Ab-Enriched category with all finite limits and colimits. This is a consequence to the fact that categories have finite limits iff it has finite products and equalizers \cite[Prop.~5.21]{Awodey}. And we know equalizers exist because equalizers of two morphisms is just the kernel of $f-g$.
	
	
	
	
	\begin{definition}[Abelian category]
		Pre-additive categories for which each mono is a kernel and each epic is a cokernel.
	\end{definition}
	Largely the purpose of abelian categories were motivated by wanting to generalize homological methods and to unify various (co)homology theories. It was defined in the modern formulation by Grothendieck in his Tohuku paper \cite{grothendieck1957quelques}. We never directly reference this paper for its mathematical content but it is interesting from a historical perspective.
	\subsubsection{Examples}
	Some examples of abelian categories are as follows,
	\begin{enumerate}
		\item \textbf{The category of modules.}
		\item\textbf{Category of representations of a group}
		\item \textbf{Category of sheaves of abelian groups on some topological space.}
		
		\begin{definition}[Presheaf]
			For a category $C$ a presheaf is any functor $F: \mathbf{C}^{\mathrm{op}}\to \mathbf{Sets}$.
		\end{definition}
		In particular in the case for a topological space $X$ a presheaf of groups (or any algebraic object) on $X$ (in truth the set of the lattice of open sets of $X$ ordered by inclusion) is a some contravariant functor $F$ which sends open sets $U \subseteq X$ to some $F(U)$, it respects inclusions (i.e. there for open sets $V \subseteq V $ is a natural transformation $\rho_{UV}: F(U)\to F(V)$ in the form of a restriction). Furthermore, function composition, unitals and empty sets going to empty sets hold (to make it a category). Note that all these notions of presheaves are really just a special case of the categorical definition where the sheaf of groups is really just a group object in the categorical presheaf.
		\begin{definition}[Sheaf of sets on a topology]
			A sheaf of a topology $X$ is a presheaf which satisfies two additional properties, for open sets $U \in X$ and open covers ${U_i}$ of $U$
			\begin{enumerate}
				\item (Locality)  \textbf{A section}, i.e. an element $s \in F(U)$ goes to zero restricted at $U_i$ for all $i$ implies $s=0$.
				\item (Gluing) If there is a collection of sections $s_i \in F(U_i)$ such that $s_i|_{U_i \cap U_j}=s_j|_{U_i \cap U_j}$ for all $i,j$ then there is some $s \in F(U)$ such that $s|_{U_i}=s_i$ for all $i$.
			\end{enumerate}
			
			These two conditions can be written is short as just saying we require $F(U)$ to be the equalizer for the following diagram
			\[\begin{tikzcd}
				{\prod_{i \in I} F(U_i)} && {\prod_{i,j}F(U_i \cap U_j)}
				\arrow[shift right=2, from=1-1, to=1-3]
				\arrow[shift left=2, from=1-1, to=1-3]
			\end{tikzcd}\]
		\end{definition}
		
		Now finally we get back to the original example. The category of sheaves of abelian groups on a topological space form a abelian category. Additivity is natural due to the functorial nature of $F$. A slightly unsatisfying proof is due to `sheafification', i.e. the left adjoint to the inclusion functor from sheaves into presheaves. Presheves of abelian groups can be understood to have all the required properties to be an Abelian category due the functorial representation. Now due to the following result \cite{stacks1} we can extend this notion to the sheaves via sheafification.
	\end{enumerate}
	
%	\subsubsection{Important results}
%	There are a few concepts and definitions relevant in the conversation of abelian categories which we will list out here for completeness. Firstly is the notion of \textbf{exact functors} the typical notion of a functor carrying forward exact sequences. With the prefix of left/right added to determine it carrying forward only left or right sides of the exact sequence.
	
	
	\begin{proposition}\label{adjointinjective}
		Given a pair of adjoint functors $F \dashv U$ between abelian categories $F:\mathbf{C} \rightleftarrows \mathbf{D}:U$ if the left adjoint $F$ is exact, faithful and if $ \mathbf{D}$ has enough injectives also $\mathbf{C}$ has enough injectives.
	\end{proposition}
	\begin{proof}
		content...
	\end{proof}
	\subsection{Derived categories}
	\subsection{Exact categories}
	\subsection{Triangulated categories}
	Add examples from Puppe sequence discussion from homological alg notes they form the triangulation in the case of the stable homotopy category.
	
	Also include quillen model cats somewhere in between.
	\end{appendices}
	
	
	\appendix
	
	
	
	
	\bibliographystyle{alpha}
	\bibliography{references}
\end{document}