\documentclass[12pt]{article}
\usepackage{graphicx}
\usepackage{amsmath}
\usepackage{fontawesome5}
\usepackage{booktabs}
\usepackage{amssymb}
\usepackage{amsthm}
\usepackage{lmodern}
\usepackage[english]{babel}
\usepackage[utf8x]{inputenc}
\usepackage[toc,page]{appendix}
\usepackage[nottoc]{tocbibind}
\numberwithin{equation}{section}
\graphicspath{ {./Images/} }
\usepackage[raggedright]{titlesec}
\usepackage{placeins}
\usepackage{tikz}
\usepackage{mathtools}
\usepackage{float}
\usepackage[autostyle]{csquotes}\usepackage{quiver}
\usepackage[activate={true,nocompatibility},final,tracking=true,kerning=true,spacing=true,factor=1100,stretch=10,shrink=10]{microtype}
\usepackage{hyperref}

\newcommand{\R}{\mathbb{R}}
\newcommand{\Q}{\mathbb{Q}}
\newcommand{\C}{\mathbb{C}}
\newcommand{\Z}{\mathbb{Z}}
\newcommand{\N}{\mathbb{N}}
\newcommand{\F}{\mathbb{F}}
\newcommand{\Hom}{{\mathrm{Hom}}}
\newcommand{\image}{{\mathrm{Im}}}
\newcommand{\kernel}{{\mathrm{Ker}}}
\newtheorem{theorem}{Theorem}[section]
\newtheorem{definition}{Definition}[section]
\newtheorem{corollary}{Corollary}[theorem]
\newtheorem{lemma}[theorem]{Lemma}

\newtheorem{proposition}{Proposition}[section]
%opening
\title{Algebraic K-Theory}
\author{Bhoris Dhanjal}
\begin{document}
	\tableofcontents
	\maketitle
	\section{Small K groups}
	The category of finitely generated projective modules is the main object of study in algebraic K-theory. This is largely motivated by the following theorem due to Swan \cite{Swan1962} which relates algebraic K-theory to topological K-theory.
	\begin{theorem}[Swan's theorem]
		There exists an equivalence of categories between $\mathrm{Vect}(X)$ the category of vector bundles over a compact, Hausdorff space $X$ and finitely generated projective $C(X)$ modules. With the cross section functor.
	\end{theorem}
	\begin{proof}
		content...
	\end{proof}
	
	\subsection{Grothendieck group $K_0$}
	The big picture idea that Grothendieck had was that of a free completion of a commutative monoid. Commutative monoids occured in nature very often as finitely generated projective modules/vector bundles. 
	
	This is a fairly natural approach which results in a Free-Forgetful adjoint pair between $\mathrm{CMon}$ and $\mathrm{Ab}$. We will refer to K book for most of the definitions \cite{weibel2013k}
		
	\begin{proposition}[Group completion functor]
		Assign $(A,+) \in \mathrm{CMon} $ to \[K_0(A) \in \mathrm{Grp}\] by taking the free group on symbols $[a]$ for $a \in A$ and quotienting the monoidal relations $[m+n]-[m]-[n]$.
	\end{proposition}
	
	The mapping is an injection iff the monoid is cancellative.
	
	\begin{proposition}[Mayer-Vietoris for group completions]
		content...
	\end{proposition}
	
	\begin{definition}[$K_0$ for a ring $A$]
		Consider the isomorphism classes of finitely generated projective modules over $A$. This forms a commutative monoid so consider its group completion $K_0(A)$
	\end{definition}
	
	\begin{proposition}[Eilenberg Swindle]
		$K_0$ for many abelian categories are trivial. If we consider $R^\infty$ as a infinitely generated free module over a ring $R$ if $P \oplus Q \equiv R^n$ then \[ P \oplus R^\infty \cong P \oplus (Q \oplus P) \oplus (Q \oplus P) \oplus \cdots \equiv (P \oplus Q) \oplus (P \oplus Q) \oplus \cdots \equiv R^\infty \] but this relation would imply $[P]=0 $ for all projectives. 
	\end{proposition}
	This extends to higher K groups with an analogue that demostrates the Quillen K space contracts, see V.1.9 in \cite{weibel2013k}.
	
	\begin{definition}[Morita equivalence for rings and ]
		content...
	\end{definition}
	
	\begin{definition}[$K_0$ for abelian category $\mathcal{A}$]
		$K_(\mathcal A)$ is generated by $[A]$ for each $A \in \mathcal{A}$ and a relation of $[A]=[A']+[A'']$ for all \[ 0 \to A' \to A \to A'' \to 0 \]
		short exact
	\end{definition}
	
	\subsection{Whitehead group $K_1$}
	\begin{definition}[Whitehead group for a ring]
		$K_1= \frac{GL(A)}{[GL(A):GL(A)]}$
		Where $GL(A)$ denotes the colimit of $GL_n(A)$ with $GL_{n}$ realized as a subgroup of $GL_{n+1}$ by placing the matrix in the top left corner. 
	\end{definition}
	\begin{proposition}
		\[ [GL(A):GL(A)]=E[A] \]
	\end{proposition}
	\begin{proof}
		content...
	\end{proof}	
	
	\begin{definition}
		$SK_1(A):= \ker \det$
		
		Where, $\det : K_1(A) \to A^*$. We have a split exact sequence
		\[ 0 \to SK_1(A) \to K_1(A) \to A^* \to 0 \]
	\end{definition}
	
	
	\begin{lemma}
		For E.D. $A$ we have $SL_n(A)=EL_n(A)$ for all $n.$
	\end{lemma}
	\begin{proof}
		content...
	\end{proof}
	
	\begin{theorem}
		$E_n(A)$ normal in $GL_n(A)$ for $n \geq 3$. 
	\end{theorem}
	\begin{proof}
		content...
	\end{proof}
	
	\begin{definition}[Unimodular row]
		For a ring $A$, an element of $A^s$ is said to be a unimodular row if its components generate the unit ideal in $A$.
	\end{definition}
	\begin{definition}[Equivalence of unimodolar rows]
		For unimodular rows $v,w\in A^s$ we say $v \sim w $ if $\exists M \in GL_s(A)$ such that $Mv=w$.
	\end{definition}
	
	\begin{proposition}
		Over a PID $A$ two unimodular rows are equivalent.
	\end{proposition}
	
	\begin{proposition}
		Over a local ring $A$ any two unimodular rows are equivalent
	\end{proposition}
	\begin{proof}
		Use the fact that projective modules over local rings are free.
	\end{proof}
	\begin{theorem}[Horrocks' theorem]
	If $R$ is a local ring then $A[x]$ then for any unimodular rows in $A^s$ with one of the elements of the row having leading coefficient 1 implies the row is equivalent to $e_1=(x,0,\cdots,0)$.
	\end{theorem}
	\subsection{$K_2$}
	
	\begin{appendices}
	\section{Projective modules}
	Recall a\textbf{ free module} of rank $n$ is one that is isomorphic to $n$ direct sums of its underlying ring. And homomorphisms from free modules to other modules are determined by the image of their generators, i.e. free objects are left adjoints to forgetful functors. \footnote{This holds in free monoids $\mathrm{Hom}_\mathbf{Mon}(F(X), M) \cong \mathrm{Hom}_\mathbf{Sets} (X, U(M))$ where $F(X)$ denotes the free monoid generated by elements from the set $X$ and $U(M)$ is the underlying set of a monoid $M$, refer to \cite[p. ~208]{Awodey} }
	
	A module $P$ is said to be \textbf{projective} if it satisfies the following lifting property, every morphism from $P$ to $N$ factors through an epi into $N$. Note that the lift need not be unique this is \textit{not} an UMP
	\[\begin{tikzcd}
		&& M \\
		\\
		P && N
		\arrow[two heads, from=1-3, to=3-3]
		\arrow[from=3-1, to=3-3]
		\arrow[dashed, from=3-1, to=1-3]
	\end{tikzcd}\]
	\begin{lemma}[Free modules are projective]
	\end{lemma}
	\begin{proof}
		Consider the preimages of images of basis of $P$ in $N$, that lie in $M$. Then map basis elements from $P$ into these preimages.
	\end{proof}
	\begin{proposition}[Equivalent definitions of projectivity]
		TFAE,
		\begin{enumerate}
			\item $P$ is projective.
			\item For all epi's between $M\twoheadrightarrow N$, the induced map $\Hom(P,g):\mathrm{Hom}(P,M) \to \mathrm{Hom}(P,N)$ sending $f \mapsto g \circ f$ for $g:M \to N$ and $f:P \to M$ is an epi.
			\item For some epi from a free module $F$ to $P$, $\mathrm{Hom}(P,F) \to \mathrm{Hom}(P,P)$ is an epi.
			\item There exists $Q$ s.t. $P \oplus Q$ is free
			\item Short exact sequences of the form $0 \to A \to B \to P \to 0$ split, i.e. isomorphic to another short exact where middle term is $A \oplus P$ \footnote{In general any epis into projective objects split (i.e. have an inverse).}
		\end{enumerate}
	\end{proposition}
	\begin{proof}
		$1 \iff 2$ is restatement of definitions.
		
		$2 \implies 3$ also just substitution.
		
		$3 \implies 4$ consider a map in the preimage of identity in $\Hom(P,P)$ which is a splitting (inverse) of the epi $F$ into $P$,
		\[\begin{tikzcd}
			& P \\
			\\
			F && P
			\arrow["g", shift left=3, two heads, from=3-1, to=3-3]
			\arrow[""{name=0, anchor=center, inner sep=0}, "f"', from=1-2, to=3-1]
			\arrow[""{name=1, anchor=center, inner sep=0}, "{\mathrm{Id}_P=g \circ f}", dashed, two heads, from=1-2, to=3-3]
			\arrow[shorten <=6pt, shorten >=6pt, Rightarrow, from=0, to=1]
		\end{tikzcd}\]
		Now we have a short exact sequence $0 \to \ker g \to F \to P \to 0$, and also $f\circ g $ is idempotent so it naturally admits a decomposition $F = \image(f \circ g) \oplus \kernel (f \circ g)$\footnote{For some idempotent $e$, $1-e$ is also an idempotent and images under these two mappings decompose any module, furthermore image of $1-e$ is just kernel of $e$}=$\image (g) \oplus \kernel (g)$ the first by the 1st isomorphism theorem and the second by $f $ being a mono.
		
		$4 \implies 2$ simply as $\hom (P \oplus Q,-) = \hom(P,-) \oplus \hom(Q,-)$
		
		$1 \iff 5$ should be clear from above.
		
	\end{proof}
	
	\begin{theorem}[Proj. fin. generated modules over local rings are free]
	\end{theorem}
	\begin{proof}
		pick a minimal set of generators and see its residue classes in $M/\mathfrak{m}M$ as the basis of it as a vector space over $R/\mathfrak{m}$.
		
		Now as for some free module $F, F=\varphi(M)\oplus K$ for some $K$ and some homomorphism $\varphi: M \to F$, (by defn of projective module), 	we get \[ M/\mathfrak{m}M \cong 	F/\mathfrak{m}F = (R/\mathfrak{m})^n\cong R^n\otimes R/\mathfrak{m} \cong F \otimes R/\mathfrak{m} \cong (\varphi(M)\oplus K) \otimes R/\mathfrak{m}\]
		
		Finally we get $M/\mathfrak{m}M \cong M/\mathfrak{m}M \oplus K/\mathfrak{m}K\implies K=\mathfrak{m}K \implies K=0$ by Nakayama
	\end{proof}
	This holds for not necessarily finitely generated modules too refer to \cite[Th.~2.5]{matsumura_1987}	.
	\begin{proposition} If $M$ is a finitely presented module over a Noetherian ring $R$ (prime ideals fin gen) then TFAE
		\begin{enumerate}
			\item $M$ is projective.
			\item $M$ localized at maximal ideals is free.
			\item A finite set of elements $\{x_i\}^n$ in $R$ generate $R$ such that $M[x_i^{-1}]$ is free over $R[x_i^{-1}]$.
		\end{enumerate}		
	\end{proposition}
	This proceeds just from the previous result.
	
	%		Note If $R$ is local and $M$ is fin-gen projective module then $M$ is free, this is a consequence of Nakayama. As $M \oplus Q = R$ so if $R$ has maximal ideal $\mathfrak{m}$ then $M/\mathfrak{m}M$ is a vector space	over the field $R/\mathfrak{m}R$ and its basis lifts to minimal set of generators of $M$, consider $N=M/\sum R m_i$ and so $ N/ IN=M/(IM+\sum_i R m_i)=M/M=0\implies N=IN$ then apply typical Nakayama to get $N=0
	%		\implies M= \sum_i R m_i$, for $I$ an ideal inside the Jacobson radical of $R$
	%		
	%		So now to prove $1 \iff 2 $ consider a finitely presented module localized over 
	%		
	%		If $M,N$ finitely presented over $R$ and their localizations are isomorphic then theres some element of $f\in R-P$ such that $M[f^{-1}] \cong N[f^{-1}]$
	\begin{theorem}[Quillen–Suslin]
		Every finitely generated projective module over a polynomial algebra is free.
	\end{theorem}
	
	
	This was an open problem for a long time as such the proof is very involved. Refer to \cite{nlab:quillen-suslin_theorem} or to a condensed proof in \cite[p. ~848]{lang02}
	
	\section{Vector bundles}
	\begin{definition}[Vector bundle]
		A real $n$ dimensional vector bundle is a triple $(E,p, B)$. Which consists of a map $p:E \to B$. Such that for all $b \in B$, $p^{-1}(b)$ has a real vector space structure. Along with the following properties,
		\begin{enumerate}
			content...
		\end{enumerate}
	\end{definition}
	
	\begin{definition}[Vector bundle mappings]
		content...
	\end{definition}
	
	\begin{definition}[Sections of a vector bundle]
		content...
	\end{definition}
	
	\begin{definition}[Pullbacks of bundles]
		content...
	\end{definition}
	
	\begin{definition}[Whitney sums]
		content...
	\end{definition}
	
	\begin{definition}[Steifel Whitney Class]
		content...
	\end{definition}
	\section{Categories}
	\subsection{Abelian Categories}
	There is a chain of conditions regarding `abelian'-ness of categories which is roughly understood as follows,
	\[ \textbf{Abelian} \subseteq \textbf{Pre-Abelian} \subseteq \textbf{Additive} \subseteq \textbf{Ab-Enriched}\]
	The motivation behind them is to have categories which resemble algebras.
	
	Ab-Enriched categories are categories such that for objects $A,B \in \mathbf{C}$ the external hom set $\Hom(A,B)$ has the structure of an abelian group, furthermore it has a well defined notion of composition (which is bilinear due to the monoidal product in Ab), $\Hom(A,B)\otimes \Hom(B,C) =\Hom(A,C)$. 
	\begin{proposition}
		In Ab-Enriched categories intial and terminal objects coincide (it is often called the zero object)
	\end{proposition}
	\begin{proof}
		Let $\mathbf{C}$ be an Ab-Enriched category. Note that the Hom-sets between objects have `zero morphisms', i.e. arrows in the Hom-set which behave like the additive identity in the Ab group induced by it. In particular for $0_{A,B}\in \Hom(A,B)$ we have the property that if $f:B \to C$ then $f\circ 0_{A,B}=0_{A,C}$ and $g: A \to D$ then $0_{A,B}\circ g=0_{D,B}$.
		
		Now suppose $0 \in \mathbf{C}$ is initial so there is a unique morphism $0\to 0$ so in its Hom-set its both the additive inverse and the identity. So for any $f:X \to 0$ we can say that by the zero morphism property $f=0$ so also $0$ is terminal.
	\end{proof}
	\begin{proposition}
		In Ab-Enriched categories finite coproducts coincide with finite products (i.e. biproducts) \footnote{This also holds over categories enriched over commutative monoids.}
	\end{proposition}
	\begin{proof}	
		Let $\mathbf{C}$ be an Ab-enriched category and $A,B\in \mathbf{C}$ consider the product $A\times B$, which is determined by the following UMP,
		\[\begin{tikzcd}
			& X \\
			A & A\times B & B
			\arrow["{p_1}"', from=2-2, to=2-1]
			\arrow["{p_2}", from=2-2, to=2-3]
			\arrow["u", dashed, from=1-2, to=2-2]
			\arrow["{x_1}"', from=1-2, to=2-1]
			\arrow["{x_2}", from=1-2, to=2-3]
		\end{tikzcd}\]
		Consider $A$ and $B$ in place of $X $ in the diagram. By the UMP we have $q_1: A \to A\times B, q_2: B \to A\times B$
		\[\begin{tikzcd}
			A && B \\
			& {A\times B} \\
			A && B
			\arrow["{p_1}"', from=2-2, to=3-1]
			\arrow["{p_2}", from=2-2, to=3-3]
			\arrow["{q_2}"', from=1-3, to=2-2]
			\arrow["{q_1}"', from=1-1, to=2-2]
			\arrow["{1_A}"', from=1-1, to=3-1]
			\arrow["{1_B}", from=1-3, to=3-3]
		\end{tikzcd}\]
		So $p_1q_1=1_A$ and $p_2q_2=1_B$ also $p_1q_2=p_2q_1=0$.
		
		Now note that $q_1p_1+q_2p_2=1_{A\times B}$ as $p_1(q_1p_1+q_2p_2)=p_1$ and $p_2(q_1p_1+q_2p_2)=p_2$. Claim this $q_1,q_2$ determine a coproduct $A +B$.
		
		We wish to show the following UMP holds for some arbitrary $C \in \mathbf{C}$
		\[\begin{tikzcd}
			A & C & B \\
			& {A\times B} \\
			A && B
			\arrow["{q_1}"', from=1-1, to=2-2]
			\arrow["{q_2}", from=1-3, to=2-2]
			\arrow["{p_1}"', from=2-2, to=3-1]
			\arrow["{p_2}", from=2-2, to=3-3]
			\arrow["{1_B}"', from=1-3, to=3-3]
			\arrow["{1_A}"', from=1-1, to=3-1]
			\arrow["{r_1}", from=1-1, to=1-2]
			\arrow["{r_2}"', from=1-3, to=1-2]
			\arrow["{f}"',dashed, from=2-2, to=1-2]
		\end{tikzcd}\]
		Define $f: A\times B \to C$ as $f=r_1p_1+r_2p_2$. Now $fq_1=r_1$ and $fq_2=r_2$ if we show uniqueness of $f$ we are done.
		
		Say $f'$ then $(f-f')1_{A \times B}=(f-f')(q_1p_1+q_2p_2)=0$. So $f=f'$.
		
		
	\end{proof}
	
	
	\begin{definition}[Additive category]
		An Ab-Enriched category which has all finite coproducts.
	\end{definition}
	
	Functors between additive categories are called \textit{additive functors}. And can be realized as functors which preserve additivity of homomorphisms between modules, $F(f+g)=F(f)+F(g).$
	
	Before proceeding further it is important to think about kernels and cokernels in the categorical sense.
	\begin{definition}[Kernel]
		A kernel is a pullback of a morphism $f:A \to B$ and the unique morphism from $0 \to B$. Provided initials and pullbacks exist.
		
		\[\begin{tikzcd}
			{\ker f} && 0 \\
			\\
			A && B
			\arrow["f", from=3-1, to=3-3]
			\arrow[from=1-3, to=3-3]
			\arrow[from=1-1, to=1-3]
			\arrow[from=1-1, to=3-1]
		\end{tikzcd}\]
	\end{definition}
	The intuition behind this definition is that alternatively it is seen as an equalizer of a function $f:A \to B$ and the unique zero morphism $0_{A,B}$. The kernel object is the part of the domain that is 'going to zero'. \footnote{A minor point to note is that in the case of Ab-Enrichments the `zero' in the Hom-sets isn't a terminal, its Hom-set specific. When you assume a Ab-Enriched category has a initial 0 however this matches up with our intuition.}
	
	
	\begin{definition}[Pre-abelian categories]
		An additive category with all morphism having kernels and cokernels.
	\end{definition}
	The above definition is equivalent to saying a pre-abelian category is a Ab-Enriched category with all finite limits and colimits. This is a consequence to the fact that categories have finite limits iff it has finite products and equalizers \cite[Prop.~5.21]{Awodey}. And we know equalizers exist because equalizers of two morphisms is just the kernel of $f-g$.
	
	
	
	
	\begin{definition}[Abelian category]
		Pre-additive categories for which each mono is a kernel and each epic is a cokernel.
	\end{definition}
	Largely the purpose of abelian categories were motivated by wanting to generalize homological methods and to unify various (co)homology theories. It was defined in the modern formulation by Grothendieck in his Tohuku paper \cite{grothendieck1957quelques}. We never directly reference this paper for its mathematical content but it is interesting from a historical perspective.
	\subsubsection{Examples}
	Some examples of abelian categories are as follows,
	\begin{enumerate}
		\item \textbf{The category of modules.}
		\item\textbf{Category of representations of a group}
		\item \textbf{Category of sheaves of abelian groups on some topological space.}
		
		\begin{definition}[Presheaf]
			For a category $C$ a presheaf is any functor $F: \mathbf{C}^{\mathrm{op}}\to \mathbf{Sets}$.
		\end{definition}
		In particular in the case for a topological space $X$ a presheaf of groups (or any algebraic object) on $X$ (in truth the set of the lattice of open sets of $X$ ordered by inclusion) is a some contravariant functor $F$ which sends open sets $U \subseteq X$ to some $F(U)$, it respects inclusions (i.e. there for open sets $V \subseteq V $ is a natural transformation $\rho_{UV}: F(U)\to F(V)$ in the form of a restriction). Furthermore, function composition, unitals and empty sets going to empty sets hold (to make it a category). Note that all these notions of presheaves are really just a special case of the categorical definition where the sheaf of groups is really just a group object in the categorical presheaf.
		\begin{definition}[Sheaf of sets on a topology]
			A sheaf of a topology $X$ is a presheaf which satisfies two additional properties, for open sets $U \in X$ and open covers ${U_i}$ of $U$
			\begin{enumerate}
				\item (Locality)  \textbf{A section}, i.e. an element $s \in F(U)$ goes to zero restricted at $U_i$ for all $i$ implies $s=0$.
				\item (Gluing) If there is a collection of sections $s_i \in F(U_i)$ such that $s_i|_{U_i \cap U_j}=s_j|_{U_i \cap U_j}$ for all $i,j$ then there is some $s \in F(U)$ such that $s|_{U_i}=s_i$ for all $i$.
			\end{enumerate}
			
			These two conditions can be written is short as just saying we require $F(U)$ to be the equalizer for the following diagram
			\[\begin{tikzcd}
				{\prod_{i \in I} F(U_i)} && {\prod_{i,j}F(U_i \cap U_j)}
				\arrow[shift right=2, from=1-1, to=1-3]
				\arrow[shift left=2, from=1-1, to=1-3]
			\end{tikzcd}\]
		\end{definition}
		
		Now finally we get back to the original example. The category of sheaves of abelian groups on a topological space form a abelian category. Additivity is natural due to the functorial nature of $F$. A slightly unsatisfying proof is due to `sheafification', i.e. the left adjoint to the inclusion functor from sheaves into presheaves. Presheves of abelian groups can be understood to have all the required properties to be an Abelian category due the functorial representation. Now due to the following result \cite{stacks1} we can extend this notion to the sheaves via sheafification.
	\end{enumerate}
	
	\subsubsection{Important results}
	There are a few concepts and definitions relevant in the conversation of abelian categories which we will list out here for completeness. Firstly is the notion of \textbf{exact functors} the typical notion of a functor carrying forward exact sequences. With the prefix of left/right added to determine it carrying forward only left or right sides of the exact sequence.
	
	
	\begin{proposition}\label{adjointinjective}
		Given a pair of adjoint functors $F \dashv U$ between abelian categories $F:\mathbf{C} \rightleftarrows \mathbf{D}:U$ if the left adjoint $F$ is exact, faithful and if $ \mathbf{D}$ has enough injectives also $\mathbf{C}$ has enough injectives.
	\end{proposition}
	\begin{proof}
		content...
	\end{proof}
	\end{appendices}
	
	
	\appendix
	
	
	
	
	\bibliographystyle{alpha}
	\bibliography{references}
\end{document}